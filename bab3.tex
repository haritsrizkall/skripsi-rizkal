%!TEX root = ./template-skripsi.tex
%-------------------------------------------------------------------------------
%                            BAB III
%               		METODOLOGI PENELITIAN
%-------------------------------------------------------------------------------

\chapter{METODOLOGI PENELITIAN}

\section{Alat dan Bahan}
	Alat dan bahan yang digunakan pada penelitian ini terbagi atas perangkat keras dan perangkat lunak yang akan dijelaskan seperti berikut.

	\subsection{Perangkat Keras}
		Pro omnium incorrupte ea. Elitr eirmod ei qui, ex partem causae disputationi nec. Amet dicant no vis, eum modo omnes quaeque ad, antiopam evertitur reprehendunt pro ut. Nulla inermis est ne. Choro insolens mel ne, eos labitur nusquam eu, nec deserunt reformidans ut. His etiam copiosae principes te, sit brute atqui definiebas id.

		\vspace{-0.5cm}

		\begin{enumerate}[a.]
		\begin{singlespace}
		\itemsep0em
			\item Kit pancar-rima IQRF TR-53B (3 unit),
			\item Kit pengunduh program CK-USB-04 (1 unit),
			\item Kit pengembangan DK-EVAL-03 (2 unit),
			\item Kit pengembangan CK-EVAL-04 (1 unit),
			\item \emph{XBee 802.15.4 Radios (Series 1)} (3 unit),
			\item \emph{XBee Explorer USB Board} (1 unit),
			\item \emph{2 channel Relay Shield For Arduino (With XBee/BTBee interface)} (2 unit),
			\item Arduino Uno (2 unit),
			\item TP-LINK MR3020 (1 unit),
			\item Kabel USB ke Serial Prolific (1 unit).
		\end{singlespace}
		\end{enumerate}

	\subsection{Perangkat Lunak}
		Pro omnium incorrupte ea. Elitr eirmod ei qui, ex partem causae disputationi nec. Amet dicant no vis, eum modo omnes quaeque ad, antiopam evertitur reprehendunt pro ut. Nulla inermis est ne. Choro insolens mel ne, eos labitur nusquam eu, nec deserunt reformidans ut. His etiam copiosae principes te, sit brute atqui definiebas id.

		\vspace{-0.5cm}

		\begin{enumerate}[a.]
		\begin{singlespace}
		\itemsep0em
			\item Arduino for Mac OS X,
			\item CoolTerm,
			\item Driver FTDI for Mac OS X,
			\item PHP, MySQL, dan uHTTPd,
			\item Python dan pustaka PySerial,
			\item IQRF IDE v 2.08 for TR-53B,
			\item SSHFS,
			\item Sublime Text 3.
		\end{singlespace}
		\end{enumerate}

\section{Alur Penelitian}
	Consul graeco signiferumque qui id, usu eu summo dicunt voluptatum, nec ne simul perpetua posidonium. Eos ea saepe prodesset signiferumque. No dolore possit est. Mei no justo intellegebat definitiones, vis ferri lorem eripuit ad. Solum tritani scribentur duo ei, his an adipisci intellegat.

\section{Tahapan Pelaksanaan}
	Consul graeco signiferumque qui id, usu eu summo dicunt voluptatum, nec ne simul perpetua posidonium. Eos ea saepe prodesset signiferumque. No dolore possit est. Mei no justo intellegebat definitiones, vis ferri lorem eripuit ad. Solum tritani scribentur duo ei, his an adipisci intellegat.

\section{Jadwal Kegiatan}
	Quo no atqui omnesque intellegat, ne nominavi argumentum quo. Eum ei purto oporteat dissentiet, soleat utamur an sit. Et assum dicam interpretaris quo. Cetero alterum ea vel, no possit alterum utroque nec. His fuisset quaestio ad. Has eu tritani incorrupte consequuntur, esse aliquip nec ne \ref{jadwal}.

	% Please remember to add \use{multirow} to your document preamble in order to suppor multirow cells
		\begin{table}[H]
		\centering
		\caption{Jadwal Penelitian.}
		\label{jadwal}
		\begin{tabular}{|c|l|l|l|l|l|l|l|}
		\hline
		\multirow{2}{*}{No} & \multirow{2}{*}{Keterangan} & \multicolumn{6}{c|}{Bulan}                                                                                                                          \\ \cline{3-8} 
		                    &                             & 1 & 2 & 3 & 4 & 5 & 6 \\ \hline
		1                   & Studi literatur                                  &\cellcolor{gray} &\cellcolor{gray}&                        &                        &                        &                         \\ \hline
		2                   & Desain                                           &                        &\cellcolor{gray}&\cellcolor{gray}&                        &                        &                         \\ \hline
		3                   & Pembelian bahan                                  &                        &                        &\cellcolor{gray}&                        &                        &                         \\ \hline
		4                   & Pembuatan prototipe                              &                        &                        &\cellcolor{gray}&\cellcolor{gray}&\cellcolor{gray}&                         \\ \hline
		5                   & Uji coba dan perbaikan                           &                        &                        &                        &\cellcolor{gray}&\cellcolor{gray}&                         \\ \hline
		6                   & Penulisan skripsi                                &                        &                        &                        &                        &                        &\cellcolor{gray}\\ \hline
		\end{tabular}
		\end{table}
	
% Baris ini digunakan untuk membantu dalam melakukan sitasi
% Karena diapit dengan comment, maka baris ini akan diabaikan
% oleh compiler LaTeX.
\begin{comment}
\bibliography{daftar-pustaka}
\end{comment}
