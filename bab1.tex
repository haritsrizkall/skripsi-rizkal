%!TEX root = ./template-skripsi.tex
%-------------------------------------------------------------------------------
% 								BAB I
% 							LATAR BELAKANG
%-------------------------------------------------------------------------------

\chapter{LATAR BELAKANG}

\section{Latar Belakang Masalah}
Saat ini, perkembangan teknologi sangatlah cepat, terlebih dengan adanya revolusi industri 4.0. Dengan perkembangan yang cepat itu kebutuhan akan perangkat lunak pun semakin banyak dan semakin kompleks. Menurut State of the Developer Nation report pada edisi 20 terdapat 24.3 juta pengembang perangkat lunak di seluruh dunia yang akan meningkat 20\% setiap tahunnya dan diperkirakan pada tahun 2030 akan ada 45 juta pengembang perangkat lunak\cite{DeveloperReport}. Hal tersebut menggambarkan betapa tingginya demand untuk pembuatan perangkat lunak. Perkembangan seperti ini membuat beberapa konsep dan cara pengembangan perangkat lunak yang sebelumnya dipakai menjadi tidak relevan karena kebutuhan perangkat lunak yang semakin banyak dan kompleks. Salah satunya adalah arsitektur monolith pada perangkat lunak, dimana sebuah perangkat lunak dibangun sebagai satu kesatuan utuh. Pada arsitektur monolith tiap bagian perangkat lunak terikat dengan erat (\emph{tightly coupled}) sehingga ketika aplikasi sudah begitu besar akan sangat sulit untuk melakukan perubahan karena ada kemungkinan perubahan tersebut berdampak pada bagian lain di perangkat lunak. Hal ini berdampak pada waktu pengembangan yang lebih lambat, karena para pengembang harus menguji dan \emph{deploy} seluruh aplikasi setiap kali terjadi perubahan.

Permasalahan yang umum terjadi pada arsitektur monolith coba diselesaikan oleh arsitektur microservices. Microservices merupakan salah satu jenis arsitektur perangkat lunak yang sedang populer belakangan ini. microservices membagi perangkat lunak menjadi beberapa bagian yang lebih kecil yang disebut servis. Masing-masing servis berjalan secara terpisah independen berdasarkan domain bisnisnya. Arsitektur microservices hadir untuk menyelesaikan masalah atau limitasi pada arsitektur tradisional monolith yang diantaranya kemudahan dalam proses-proses maintainability, reusability, scalability, availability, dan automated deployment.

Meskipun arsitektur miroservices menyelesaikan banyak permasalahan yang ada pada arsitektur monolith, terdapat masalah dan tantangan baru pada bagian monitoring dan memahami keseluruhan sistem secara utuh. Dikarenakan jumlah servis yang banyak dan tiap servis terdistribusi secara independen, cukup sulit untuk melacak aliran data dan mengetahui bagian yang bermasalah atau menjadi \emph{bottle-neck} pada sistem. Karena hal tersebutlah diibutuhkan suatu cara untuk menyelesaikan permasalahan tersebut. Sebagai gambaran, perusahaan streaming online Netflix memiliki lebih dari 1000 microservice\cite{NetflixMicroservices}, sebuah request yang masuk dapat melewati banyak servis sebelum memberikan respon. Jika terjadi masalah pada salah satu servis, maka akan sulit untuk mengetahui bagian mana yang bermasalah karena tidak mungkin seorang pengembang dapat mengetahui perilaku sistem secara keseluruhan. Oleh karena itu, perlu adanya distributed tracing pada sistem microservices.

Distributed tracing merupakan suatu teknik pada sistem perangkat lunak yang digunakan untuk memantau dan menganalisa aliran sebuah request yang melewati beberapa komponen dalam hal microservices merupakan servis. Distributed tracing membuat pengembang untuk memahami perilaku dan kinerja sistem secara keseluruhan.

Pada penelitian kali ini, akan dilakukan implementasi dan analisis distributed tracing pada sistem microservices. Hasil dari penelitian ini diharapkan dapat dijadikan acuan dan pertimbangan dalam melakukan implementasi distributed tracing pada sistem microservices.

\section{Rumusan Masalah}
Berdasarkan latar belakang yang telah dituliskan sebelumnya, terdapat masalah dan tantangan dalam upaya mengetahui keadaan internal suatu sistem perangkat lunak berbasis microservices. Masalah tersebut muncul karena arsitektur microservices membagi sistem menjadi bagian-bagian yang lebih kecil yang secara alamiah akan membuat sistem perangkat lunak menjadi lebih kompleks. Permasalahan yang muncul pada arsitektur microservices terkait upaya mengetahui keadaan internal sistem terebut diantaranya adalah:
\begin{enumerate}
\item Apakah metode distributed tracing dapat mengetahui aliran request pada sebuah sistem microservices?
\item Apakah metode distributed tracing dapat membantu dalam deteksi error dan bottle-neck pada sebuah sistem microservices?
\item Bagaimana menerapkan distributed tracing pada sistem microservices yang memiliki teknologi yang berbeda-beda?
\item Bagaimana distributed tracing dapat membantu dalam proses monitoring dan debugging pada sistem microservices?
\item Bagaimana dampak dari penerapan distributed tracing terhadap performa dari sistem microservices?
\end{enumerate}

\section{Tujuan Penelitian}
Penelitian ini bertujuan untuk mengetahui efektivitas dari implementasi distributed tracing dalam deteksi error dan bottle-neck pada sistem microservices serta dampak implementasi tersebut terhadap kesluruhan performa sistem.

\section{Batasan Penelitian}
Batasan penelitian ini adalah:
\begin{enumerate}
\item Fokus penelitian ini adalah pada penerapan distributed tracing pada sistem microservices dalam mendeteksi error yang terjadi pada sistem microservices serta dampak dari penerapan distributed tracing terhadap performa dari sistem microservices.
\item Sistem microservices yang digunakan menggunakan dua bahasa pemrograman yang berbeda yaitu Javascript dan Go.
\item Sistem microservices yang digunakan menggunakan dua sistem basis data yang berbeda yaitu Postgresql dan MySQL.
\item Semua tools yang digunakan pada penelitian ini merupakan open source.
\item Sistem microservices yang digunakan pada penelitian ini berjalan pada kubernetes dengan 1 node master dan 2 node worker.
\end{enumerate}


% \section{Batasan Masalah}
% Batasan masalah pada penelitian ini adalah:
% \begin{enumerate}
% \item Objek penelitian: Penerapan dan analisis distributed tracing pada sistem microservices
% \item Metode penelitian: Penerapan distributed tracing dilakukan dengan melakukan eksperimen pada sebuah sistem microservices yang memiliki teknologi yang berbeda-beda. Hasil dari penelitian ini adalah mengetahui efektivitas dari penerapan distributed tracing pada sistem microservices serta dampak dari penerapan distributed tracing terhadap performa dari sistem microservices.
% \item Waktu dan tempat penelitian: Waktu penelitian dari February 2023 – Juni 2023, di rumah peneliti
% \item Variabel: Variabel bebas pada penelitian ini adalah metode dan teknologi yang digunakan untuk penerapan distributed tracing. Sementara itu, variabel terikat pada penelitian ini adalah efektivitas dari penerapan distributed tracing pada sistem microservices serta dampak dari penerapan distributed tracing terhadap performa dari sistem microservices.
% \item Hipotesis: Distributed tracing dapat dengan efektif mendeteksi kesalahan serta anomali pada sistem microservices.
% \item Keterbatasan penelitian:
% \begin{itemize}
% \item Sistem microservices yang digunakan pada penelitian memiliki jumlah 6 service dengan dua bahasa pemrograman yaitu Go dan NodeJs.
% \item Sistem microservices yang digunakan pada penelitian ini menggunakan 2 database yaitu Postgresql dan MySQL.
% \item Semua servis yang digunakan pada penelitian ini berjalan pada docker container.
% \end{itemize}
% \end{enumerate}

\section{Manfaat Penelitian}
Manfaat dari penelitian ini adalah:
\begin{enumerate}
\item Bagi praktisi pengembang perangkat lunak, penelitian ini dapat dijadikan dasar untuk pembuatan tools monitoring untuk mengetahui keadaan internal dari sistem microservices.
\item Bagi penulis, penelitian ini dapan menambah wawasan, ilmu dan pengetahuan dalam pembuatan tulisan ilmiah, khususnya pada topik terkait monitoring pada sistem perangkat lunak modern.
\item Bagi pelaku bisnis, penerapan observability dapat meningkatkan customer experience.\\
\end{enumerate}

\section{Sistematika Penulisan}
\noindent
\textbf{BAB I : PENDAHULUAN}

Pada bab ini dijelaskan latar belakang, rumusan masalah, batasan, tujuan, manfaat, keaslian penelitian, dan sistematika penulisan.\\

\noindent
\textbf{BAB II : TINJAUAN PUSTAKA DAN LANDASAN TEORI}
Bab tersebut berisi tentang hasil-hasil penelitian yang telah dilakukan oleh para peneliti sebelumnya yang dapat ditemukan dalam tinjauan pustaka, serta teori-teori yang mendukung penelitian yang terdapat pada dasar teori dan analisis perbandingan metode dari penelitian sebelumnya.\\

\noindent
\textbf{BAB III : METODOLOGI PENELITIAN}

Bab tersebut membahas mengenai rincian mengenai alat dan bahan yang digunakan dalam penelitian, seperti perangkat keras dan perangkat lunak yang digunakan. Selain itu, dijelaskan pula mengenai metode yang digunakan dalam penelitian. Bab tersebut juga membahas mengenai alur penelitian, yaitu tahapan-tahapan yang dilakukan dalam penelitian untuk mencapai tujuan yang telah ditetapkan.\\

\noindent
\textbf{BAB IV : HASIL DAN PEMBAHASAN}

Pada bab ini dijelaskan hasil penelitian dan pembahasannya.\\

\noindent
\textbf{BAB V : KESIMPULAN DAN SARAN}

Bab ini berisi kesimpulan yang diperoleh dari penelitian yang telah dilakukan. Selain itu, bab ini juga memuat saran-saran untuk pengembangan penelitian selanjutnya berdasarkan hasil temuan yang didapatkan dari penelitian yang telah dilakukan.\\

% Baris ini digunakan untuk membantu dalam melakukan sitasi
% Karena diapit dengan comment, maka baris ini akan diabaikan
% oleh compiler LaTeX.
\begin{comment}
\bibliography{daftar-pustaka}
\end{comment}
